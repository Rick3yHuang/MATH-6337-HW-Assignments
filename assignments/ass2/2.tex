\section{Question 2}

\begin{question}
    Read subsection 1.1 in the textbook (pp 15-21). Note in particular the definition of a metric space, and the corresponding definitions for matric spaces of notions which were discussed in class for normed spaces.
\end{question}

\subsection{Part i}

\begin{question}
   Prove that every normed space is a metric space.
\end{question}

\begin{answer}
    \begin{proof}
        Let X be a normed space, we could define a function on $X \times X$ by $(a, b) \rightarrow\|a-b\|$. We claim this is a metric space.

        First, since $X$ is a normed space, then $d(x,y) = \lVert x - y \rVert geq 0$, and $d(x,y) = \lVert x - y \rVert = 0$ if and only if $x -y = 0$, which means $x = y$.
        
        Second, $d(x,y) = \lVert x - y \rVert = \lVert (-1)\cdot (y-x) \rVert = \lvert -1 \rvert \lVert y - x \rVert = \lVert y - x \rVert = d(y,x)$, by the homogenuous property of the normed space..
        
        Finally, $d(x,y) = \lVert x - y \rVert = \lVert (x-z)-(z-y)\rVert \leq \lVert x - z \rVert + \lVert z - y \rVert = d(x,z)+ d(z,y)$ by the triangle inequality of the normed space.
    \end{proof}
\end{answer}

\subsection{Part ii}

\begin{question}
    Give 3 different examples of metric spaces which are not normed spaces (note, in particular, that a metric space need not necessarily be a vector space).
\end{question}

\begin{answer}
    Example 1: Let $X = \{x_1,x_2, \cdots, x_n\}$ for some $n$, then let $x,y \in X$, the metric:
    \begin{equation}
        d(x,y) = 
        \begin{cases}
            0 & x = y\\
            1 & \text{Otherwise.}
        \end{cases}
    \end{equation}
    defines a metric space but not a normed space since the homogenuous property is not satisfied.
    
    For the same reason, we have the example 2.
    
    Example 2: 
    Let $X = \mathbb{R}$, then let $x,y \in X$, the metric:
    \begin{equation}
        d(x,y) = \left|e^x-e^y\right|
    \end{equation}
    defines a metric space but not a normed space.
    
    Also because the homogenuous property doesn't hold, the example 3 is a metric space but not a normed space with the following metric:
    
    Let $X = \mathbb{R}$, then let $x,y \in X$, define $d(x, y)=\frac{|x-y|}{1+|x-y|}$
\end{answer}

\subsection{Part iii}

\begin{question}
   Let $B$ be a Banach space and $E \subset B$. Prove that $E$ is closed in $B$ if and only if it is a complete metric space with the metric induced by the norm of $B$
\end{question}

\begin{answer}
    \begin{proof}
        First assume E be a closed, let $\{x_n\} \subseteq E$ be a Cauchy sequence, then since $E \subset B$, $\{x_n\} \subseteq B$. Because $B$ is a Banach space, then $x_n \xrightarrow{n \to \infty} x$ for some $x \in B$. Since $E$ is closed, then $x \in E$. Thus, $E$ is a complete metric space.
    
        Next assume $E$ is a complete metric space. Let $\{x_n\} \subseteq E$ be a convergent sequence, so we know $\{x_n\}$ is Cauchy. Then since $E$ is a complete metric space, $\exists x \in E \; s/t \; x_n \xrightarrow{n \to \infty} x$. Thus, every convergent sequence in $E$ converge to an element in $E$. This implies that $E$ is closed.
    \end{proof}
\end{answer}
